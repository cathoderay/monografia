\begin{titlepage}
\begin{center}
  \Large
  \textbf{Abstract}
\end{center}
Kaiser, R. A. Aplicação do algoritmo \textit{Perceptually Important Points} em séries temporais de datacenters.\\

The main purpose of this work is helping operators of datacenters in the task of visualizing the behaviour of their devices and services through time, represented by large time series. In order to accomplish that, a technique used in pattern recognition from the financial market context was choosed. The ``Perceptually Important Points'' algorithm gives a method for dimensionality reduction and a mechanism to automatically extract the most important points from a human observer perspective, favouring compression and a good visualization of time series with high dimensionality. The implementation of the algorithm and its integration in an existing monitoring system was explored and encompasses the content of this work.\\

\textbf{Keywords:} Time series, Dimensionality Reduction, Perceptually Important Points, Datacenters, Clojure.

\end{titlepage}
