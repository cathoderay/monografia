\begin{titlepage}
\begin{center}
  \Large
  \textbf{Resumo}
\end{center}
Kaiser, R. A. Aplicação do algoritmo \textit{Perceptually Important Points} em séries temporais de datacenters.\\

O propósito deste trabalho é auxiliar operadores de datacenters na visualização de gráficos de séries temporais, que representam a evolução do comportamento de seus equipamentos e serviços. Para tanto, foi buscada inspiração em um algoritmo utilizado para o reconhecimento de padrões em séries temporais do mercado financeiro. O algoritmo \textit{Perceptually Important Points} ao mesmo tempo que contribui para a redução de dimensionalidade, fornece um mecanismo automatizado de eleição dos pontos mais importantes de uma série temporal para um observador humano, favorecendo, deste modo, sua visualização e tráfego pela rede. A implementação do algoritmo e sua implantação em um sistema de monitoração de datacenters já existente, hoje utilizado por grandes datacenters do Brasil, constituem o cerne deste trabalho. \\

\textbf{Palavras-Chave:} Séries Temporais, Redução de dimensionalidade, Perceptually Important Points, Datacenters, Clojure.

\end{titlepage}
