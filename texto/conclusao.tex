\chapter{Conclusão}
\label{chap:conclusao}
Com o aumento da capacidade de processamento e armazenamento das máquinas seguindo a Lei de Moore, a possibilidade de se lidar com séries temporais cada vez maiores vem crescendo ao longo dos anos. A literatura existente sobre o assunto é extremamente vasta e rica, com aplicações nos mais variados campos do conhecimento humano.

O algoritmo escolhido para esse trabalho tem sido usado em séries temporais do mercado financeiro e aplicações no domínio de datacenters é nova até onde foi possível se investigar.

A relevância deste trabalho se manifesta a partir da aprovação por parte dos usuários do sistema HOLMES, com relatos confirmando a diminuição no tempo de identificação e solução de problemas em ambientes de operação de datacenters.

Apesar dos objetivos definidos para este trabalho -- expostos na seção \ref{sec:objetivos} -- terem sido alcançados com êxito, muitas melhorias no software ainda se fazem necessárias; como um mecanismo para o cálculo de erro e escolha automatizada do número mínimo de pontos -- descritos na seção \ref{sec:calculo-erro} -- para representar a série temporal original, não incluída na versão final da implementação devido o fator tempo. 

Por fim, pode-se endereçar a trabalhos futuros, a integração com outros algoritmos de representação e redução de dimensionalidade na tentativa de se reduzir as limitações do algoritmo \textit{Perceptually Important Points}, além de testes comparativos mais extensos no domínio de datacenters.
